\title{LMMP: Stereo Module Testing Plan}
\author{
        Ara V. Nefian \and Michael Broxton
}
\date{\today}

\documentclass[12pt]{article}

\begin{document}
\maketitle

This document describes a testing plan for the LMMP stereo module.
{\underline {\bf Accuracy measures}}

\begin{itemize}
\item {\bf Option 1:} Short term, if ground truth data (Apollo Panoramic images) is available (Feb 15). Let the predicted horizontal and vertical
      disparity map for the $k$th image be denoted as ${\bf P}_{k, h}(i,j)$ and ${\bf P}_{k, v}(i,j)$  respectively with the corresponding ground truth disparity map ${\bf T}_{k, h}(i,j)$ and ${\bf T}_{k, v}(i,j)$ for all pixels $(i,j)$. 

The ${\bf average}$ ${\bf error}$ is 
      ${\bf E} =\frac{1}{K}\sum_k\frac{\sum_{ij} (P_{k,v}(i,j)-T_{k,v}(i,j))^2 + (P_{k,h}(i,j)-T_{k,h}(i,j))^2}{N_{k,pred}} $ where $N_{k,pred}$ 
      is the number of pixels for which the disparity map was computed in the $k$th image, and $K$ is the total number of images for 
      which we have ground truth. 

      The ${\bf average}$ ${\bf coverage}$ ${\bf C}$ measures the number of pixels for which the disparity is determined: 
${\bf C} = \frac{1}{K}\sum_k\frac{N_{k,pred}}{N_{k,true}}$, where $N_{k, true}$ is the resolution of the ground truth images.
  
     The goal is to have a small average error ${\bf E}$ and large average coverage ${\bf C}$ measures.

\item {\bf Option 2:} Very short term, before the ground truth data is made available (Jan 15th).
                    Compute the ${\bf E}$ measure using existing synthetic ground truth data, and the ${\bf C}$ 
                    measure using the MOC and Apollo Metric data.

\item {\bf Option 3:} On the longer term the ${\bf E}$ measure will be weighted by the confidence score of the Bayesian subpixel correlator.

 
       ${\bf \tilde E} =\frac{1}{K}\sum_k\frac{\sum_{ij} ((P_{k,v}(i,j)-T_{k,v}(i,j))^2 + (P_{k,h}(i,j)-T_{k,h}(i,j))^2)\frac{p_{ij}}{\sum_{ij}p_{ij}}}{N_{k,pred}} $      
\end{itemize}

{\underline {\bf Performance measures}}
 Very short term (Jan 15th). Compute the number of additions, multiplications log and exp operations per pixel and run time required to generate the dense disparity map.
%\begin{abstract}
%This is the paper's abstract \ldots
%\end{abstract}

%\section{Introduction}
%This is time for all good men to come to the aid of their party!

%\paragraph{Outline}
%The remainder of this article is organized as follows.
%Section~\ref{previous work} gives account of previous work.
%Our new and exciting results are described in Section~\ref{results}.
%Finally, Section~\ref{conclusions} gives the conclusions.

%\section{Previous work}\label{previous work}
%A much longer \LaTeXe{} example was written by Gil~\cite{Gil:02}.

%\section{Results}\label{results}
%In this section we describe the results.

%\section{Conclusions}\label{conclusions}
%We worked hard, and achieved very little.

\bibliographystyle{abbrv}
\bibliography{simple}

\end{document}
This is never printed
