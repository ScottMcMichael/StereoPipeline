\chapter{Installation}

\section{Binary Installation}

This is the recommended method. Only the Stereo Pipeline binaries
are required. ISIS is required only for users who wish to process
NASA non-terrestrial imagery.  A full ISIS installation is not
required for operation of Stereo Pipeline programs (only the ISIS
data directory is needed), but is required for certain preprocessing
steps before Stereo Pipeline programs are run for planetary data.
If you only want to process terrestrial Digital Globe imagery, skip
to section \ref{quickstartDG}.

\begin{description}
\item [{Stereo~Pipeline~Tarball}] \hspace*{\fill} \\
The main Stereo Pipeline page is
\url{http://irg.arc.nasa.gov/ngt/stereo}.  Download the option that
matches the platform you wish to use. The recommended, but
optional, \ac{ISIS} version is listed next to the name.

\item [{USGS~ISIS}] \hspace*{\fill} \\
If you are working with non-terrestrial imagery, you will need to install
\ac{ISIS} so that you can perform preprocessing such as radiometric
calibration and ephemeris attachment. The \ac{ISIS} installation guide is at
\url{http://isis.astrogeology.usgs.gov/documents/InstallGuide}.  You
must use their binaries as-is; if you need to recompile, you can
follow the \emph{Source Installation} guide for the Stereo Pipeline in
Section~\ref{sec:Source-Installation}.  Note also that the \ac{USGS}
provides only the current version of \ac{ISIS} and the previous
version (denoted with a `\texttt{\_OLD}' suffix) via their
\texttt{rsync} service. If the current version is newer than the
version of ISIS that the Stereo Pipeline is compiled against, be
assured that we're working on rolling out a new version. However,
since Stereo Pipeline has its own self-contained version of ISIS's
libraries built internally, you should be able to use a newer version
of ISIS with the now dated version of \ac{ASP}. This is assuming no major
changes have taken place in the data formats or camera models by
\ac{USGS}. At the very least, you should be able to rsync the previous
version of ISIS if a break is found.  To do so, view the listing of
modules that is provided via the
`\texttt{rsync~isisdist.astrogeology.usgs.gov::}' command.  You should
see several modules listed with the `\texttt{\_OLD}' suffix.  Select
the one that is appropriate for your system, and \texttt{rsync}
according to the instructions.

In closing, running the Stereo Pipeline executables only requires that
you have downloaded the ISIS secondary data and have appropriately set
the \texttt{ISIS3DATA} environment variable. This is normally
performed for the user by ISIS startup script,
\texttt{\$ISISROOT/scripts/isis3Startup.sh}.

\end{description}

\subsection{Quick Start for ISIS Users}
\begin{description}

\item[{Fetch~Stereo~Pipeline}] ~\\
Download the Stereo Pipeline from \url{http://irg.arc.nasa.gov/ngt/stereo}.

\item [{Fetch~ISIS~Binaries}] ~\\
As detailed at \url{http://isis.astrogeology.usgs.gov/documents/InstallGuide}.

\item [{Fetch~ISIS~Data}] ~\\
As detailed at \url{http://isis.astrogeology.usgs.gov/documents/InstallGuide}.

\item [{Untar~Stereo~Pipeline}] ~\\
\texttt{tar xzvf StereoPipeline-\textit{VERSION-ARCH-OS}.tar.gz}

% Verbatim has way too much white space. Couldn't seem to take care of it with
% vskip/vspace negative. Sigh.
\item [{Add~Stereo~Pipeline~to~Path~(optional)}] ~\\
bash: \texttt{export PATH="\textit{/path/to/StereoPipeline}/bin:\$\{PATH\}"} \\
csh:  \texttt{setenv PATH "\textit{/path/to/StereoPipeline}/bin:\$\{PATH\}"}

\item[Set~Up~ISIS] ~\\
bash: \\
\hspace*{2em}\texttt{export ISISROOT=\textit{/path/to/isisroot}} \\
\hspace*{2em}\texttt{source \$ISISROOT/scripts/isis3Startup.sh} \\
csh: \\
\hspace*{2em}\texttt{setenv ISISROOT \textit{/path/to/isisroot}} \\
\hspace*{2em}\texttt{source \$ISISROOT/scripts/isis3Startup.csh}

\item [{Try~It~Out}] ~\\
See Chapter~\ref{ch:moc_tutorial} for an example.
\end{description}

\subsection{Quick Start for Digital Globe Users}
\label{quickstartDG}
\begin{description}

\item[{Fetch~Stereo~Pipeline}] ~\\
Download the Stereo Pipeline from \url{http://irg.arc.nasa.gov/ngt/stereo}.

\item[{Untar~Stereo~Pipeline}] ~\\
\texttt{tar xvfz StereoPipeline-\textit{VERSION-ARCH-OS}.tar.gz}

\item [{Try~It~Out}] ~\\
Processing Earth imagery is described in the data processing tutorial
in chapter \ref{ch:dg_tutorial}.

\end{description}

\subsection{Quick Start for Aerial and Historical Imagery}
\label{quickstartAerial}

Fetch the software as before. Processing imagery without accurate camera
pose information is described in chapter \ref{ch:sfm}.

\subsection{Common Errors}

Here are some errors you might see, and what it could mean. Treat
these as templates for problems.  In practice, the error messages might
be slightly different.

\begin{verbatim}
**I/O ERROR** Unable to open [$ISIS3DATA/Some/Path/Here].
Stereo step 0: Preprocessing failed
\end{verbatim}

You need to set up your ISIS environment or manually set the correct
location for \texttt{ISIS3DATA}.

\begin{verbatim}
point2mesh stereo-output-PC.tif stereo-output-L.tif
[...]
99%  Vertices:   [************************************************************] Complete!
       > size: 82212 vertices
Drawing Triangle Strips
Attaching Texture Data
zsh: bus error  point2mesh stereo-output-PC.tif stereo-output-L.tif
\end{verbatim}

The source of this problem is an old version of OpenSceneGraph in
your library path. Check your \verb#LD_LIBRARY_PATH# (for Linux),
\verb#DYLD_LIBRARY_PATH# (for OSX), or your \verb#DYLD_FALLBACK_LIBRARY_PATH#
(for OSX) to see if you have an old version listed, and remove it
from the path if that is the case. It is not necessary to remove the
old versions from your computer, you just need to remove the reference
to them from your library path.

\begin{verbatim}
bash: stereo: command not found
\end{verbatim}

You need to add the \texttt{bin} directory of your deployed Stereo
Pipeline installation to the environmental variable \texttt{PATH}.

\section{\label{sec:Source-Installation}Installation from Source}

This method is for advanced users. You will need to fetch the Stereo Pipeline source code from GitHub at
\url{https://github.com/NeoGeographyToolkit/StereoPipeline} and then
follow the instructions specified in \texttt{INSTALLGUIDE}.

\section{\label{sec:Settings}Settings Optimization}

Finally, the last thing to be done for Stereo Pipeline is to setup up
Vision Workbench's render and logging settings. This step is optional,
but for best performance some thought should be applied here.

Vision Workbench is a multi-threaded image processing library used by
Stereo Pipeline. The settings by which Vision Workbench processes is
configurable by having a \texttt{.vwrc} file hidden in your home
directory. Below is an example.

\newpage
\begin{minipage}{0.94\linewidth}
\small\listinginput{1}{LogConf.example}
\end{minipage}

There are a lot of possible options that can be implemented in the
above example. Let's cover the most important options and the concerns
the user should have when selecting a value.

\subsection{Performance Settings}

\begin{description}

\item[default\_num\_threads \textnormal (default=2)] \hfill \\
This sets the maximum number of threads that can be used for
rendering. When stereo's \texttt{subpixel\_rfne} is running you'll
probably notice 10 threads are running when you have
\texttt{default\_num\_threads} set to 8. This is not an error, you are
seeing 8 threads being used for rendering, 1 thread for holding
\texttt{main()}'s execution, and finally 1 optional thread acting as the
interface to the file driver.

It is usually best to set this parameter equal to the number of
processors on your system. Be sure to include the number of logical
processors in your arithmetic if your system supports hyper-threading.
Adding more threads for rasterization increases the memory demands of
Stereo Pipeline. If your system is memory limited, it might be best to
lower the \texttt{default\_num\_threads} option.

\item[write\_pool\_size \textnormal (default=21)] \hfill \\
The \texttt{write\_pool\_size} option represents the max waiting pool
size of tiles waiting to be written to disk. Most file formats do not
allow tiles to be written arbitrarily out of order. Most however will
let rows of tiles to be written out of order, while tiles inside a row
must be written in order. Because of the previous constraint, after a
tile is rasterized it might spend  some time waiting in the `write
pool' before it can be written to disk. If the `write pool' fills up,
only the next tile in order can be rasterized. That makes Stereo
Pipeline perform like it is only using a single processor.

Increasing the \texttt{write\_pool\_size} makes Stereo Pipeline more
able to use all processing cores in the system. Having this value too
large can mean excessive use of memory as it must keep more portions
of the image around in memory while they wait to be written. This 
number should be larger than the number of threads, perhaps by
about 20.

\item[system\_cache\_size \textnormal (default=805306368)] \hfill \\
Accessing a file from the hard drive can be very slow. It is especially
bad if an application needs to make multiple passes over an input
file. To increase performance, Vision Workbench will usually leave an
input file stored in memory for quick access. This file storage is
known as the 'system cache' and its max size is dictated by
\texttt{system\_cache\_size}. The default value is 768 MB.

Setting this value too high can cause your application to crash. It is
usually recommend to keep this value around 1/4 of the maximum
available memory on the system. The units of this property is in
bytes.

The reccomendations for these values are based on use of the block
matching algorithm in ASP.  When using memory intensive algorithms
such as SGM you may wish to lower some of these values (such as
the cache size) to leave more memory available for the algorithm to use.

\end{description}

\subsection{Logging Settings}
\label{logging}

The messages displayed in the console by Stereo Pipeline are grouped
into several namespaces, and by level of verbosity. An example of
customizing Stereo Pipeline's output is given in the \texttt{.vwrc} file
shown above.

Several of the tools in Stereo Pipeline, including \texttt{stereo},
automatically append the information displayed in the console to a log
file in the current output directory. These logs contain in addition some
data about your system and settings, which may be helpful in resolving
problems with the tools.

It is also possible to specify a global log file to which all tools will
append to, as illustrated in \texttt{.vwrc}.
